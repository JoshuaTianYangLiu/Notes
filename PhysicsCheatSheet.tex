\documentclass[10pt,landscape]{article}
\usepackage{multicol}
\usepackage{calc}
\usepackage{ifthen}
\usepackage[landscape]{geometry}
\usepackage{hyperref}
\usepackage{blindtext}
\usepackage{amsmath}
\usepackage{mathtools}

\DeclarePairedDelimiter\abs{\lvert}{\rvert}%
\newcommand{\ve}[1]{\overrightarrow{#1}}
\newcommand{\lam}{\lambda}
\newcommand{\ti}{\times}



%Credit to https://github.com/wch/latexsheet/blob/gh-pages/latexsheet.tex
% This sets page margins to .5 inch if using letter paper, and to 1cm
% if using A4 paper. (This probably isn't strictly necessary.)
% If using another size paper, use default 1cm margins.
\ifthenelse{\lengthtest { \paperwidth = 11in}}
	{ \geometry{top=.5in,left=.5in,right=.5in,bottom=.5in} }
	{\ifthenelse{ \lengthtest{ \paperwidth = 297mm}}
		{\geometry{top=1cm,left=1cm,right=1cm,bottom=1cm} }
		{\geometry{top=1cm,left=1cm,right=1cm,bottom=1cm} }
    }
    

\pagestyle{empty}

\makeatletter
\renewcommand{\section}{\@startsection{section}{1}{0mm}%
                                {-1ex plus -.5ex minus -.2ex}%
                                {0.5ex plus .2ex}%x
                                {\normalfont\large\bfseries}}
\renewcommand{\subsection}{\@startsection{subsection}{2}{0mm}%
                                {-1explus -.5ex minus -.2ex}%
                                {0.5ex plus .2ex}%
                                {\normalfont\normalsize\bfseries}}
\renewcommand{\subsubsection}{\@startsection{subsubsection}{3}{0mm}%
                                {-1ex plus -.5ex minus -.2ex}%
                                {1ex plus .2ex}%
                                {\normalfont\small\bfseries}}
\makeatother


% Don't print section numbers
\setcounter{secnumdepth}{0}


\setlength{\parindent}{0pt}
\setlength{\parskip}{0pt plus 0.5ex}


\setlength{\premulticols}{1pt}
\setlength{\postmulticols}{1pt}
\setlength{\multicolsep}{1pt}
\setlength{\columnsep}{2pt}

\begin{document}
\raggedright
\footnotesize
\begin{center}
    \Large{\textbf{Physics Cheat Sheet}} \\
\end{center}
\begin{multicols}{3}
    \setlength{\parindent}{0cm}
    \section{Kinematics}
    Position $\ne$ Displacement

    $\ve{d_1}=20m[E]$

    $\Delta \ve{d}=\ve{d}_{final}-\ve{d}_{initial}$

    $\Delta \ve{d}_{total}=\Delta\ve{d}_{1}+\Delta\ve{d}_{2}$

    $\Delta \ve{d}=\ve{v}\ti \Delta t$

    $\ve{a}=\frac{\Delta \ve{v}}{\Delta \ve{t}}=\frac{\ve{v}_f-\ve{v}_i}{t_2-t_1}$

    $\ve{v}_f=\ve{v}_i+\ve{a}\ti \Delta t$

    % \section{Comparing Graphs of Linear Motion}
    \section{Five Key Equations}
    \begin{align*}
        \ve{a}&=\frac{\ve{v}_f-\ve{v}_i}{\Delta t} & \Delta \ve{d}\\
        \Delta \ve{d}&=\frac{\ve{v}_f+\ve{v}_i}{2}\Delta t&\ve{a}\\
        \Delta \ve{d}&=\ve{v}_1\Delta t+\frac{1}{2}\ve{a}\Delta t^2&\ve{v}_2 \\
        \Delta \ve{d}&=\ve{v}_2\Delta t-\frac{1}{2}\ve{a}\Delta t^2&\ve{v}_1 \\
        \ve{v}_2^2&=\ve{v}_1^2+2\ve{a}\Delta \ve{d}&\Delta t\\
    \end{align*}
    
    $g=-9.8$ $m/s^2$

    \section{Adding Vectors}
    $\Delta \ve{d}=n$ m$[E\theta^{\circ}N]$

    $\Delta \ve{d}_x=n\cos\theta$

    $\Delta \ve{d}_y=n\sin\theta$

    $\Delta \ve{d}_T^2=\Delta \ve{d}_1+\Delta\ve{d}_2$

    $\Delta \ve{d}_{T_x}=\Delta\ve{d}_{1_x}+\Delta\ve{d}_{2_x}$

    $\Delta \ve{d}_{T_y}=\Delta\ve{d}_{1_y}+\Delta\ve{d}_{2_y}$

    $\abs{\Delta \ve{d}_T}=\sqrt{(\Delta \ve{d}_{1_x}-\Delta \ve{d}_{2_x})^2+(\Delta \ve{d}_{1_y}-\Delta \ve{d}_{2_y})^2}$

    $\theta=\tan(\frac{\Delta \ve{d}_y}{\Delta \ve{d}_x})$

    % \section{Projectile Motion}
    \section{Acceleration in Two Dimensions}

    $\ve{v}_{og}=\ve{v}_{om}+\ve{v}_{mg}$

    $1$ N $=1$ kg$\frac{m}{s^2}$

    $\ve{F}_g=m\ve{g}$

    $\ve{F}_f\propto F_N$

    $\mu=\frac{F_f}{F_N}$

    \section{Newtons Laws}

    $F_{net_y}=F_N-F_g$

    $F_{net_x}=F_a-F_f$

    $\ve{F}_{net}=m\ti\ve{a}$

    $\abs{\ve{a}}=\frac{\abs{\ve{F}_{net}}}{m}$

    $\ve{F_{Action}}=-\ve{F_{Reaction}}$

    % \section{Applications To Newtons Laws}
    \section{Work and Energy}

    $1$ J $=1$ kg$\frac{m^2}{s^2}=1$ N$\ti m$

    $W=\ve{F_{app}}\ti\Delta\ve{d}$

    $W=F_a\cos\theta\ti\Delta d$

    $W=\frac{F_{max}\ti\Delta d}{2}$

    $W=F_{av}\ti\Delta d$

    $E_k=\frac{mv^2}{2}$

    $W=\Delta E_k$

    $E_g=mgh$

    $E_{T_1}=E_{T_2}$

    $E_T=E_k+E_g$

    $E_{k_1}+E_{g_1}=E_{k_2}+E_{g_2}$

    $E_{T_1}+W_{done}=E_{T_2}$

    $W_{done}=\Delta E_T$

    $P=\frac{W_{net}}{\Delta t}$

    $\text{Efficiency}=\frac{\text{useful Output Energy}}{\text{Total input energy}}\ti 100\%$

    $E=\frac{E_{output}}{E_{input}}\ti 100\%$

    % \section{Thermal Energy and Heat Transfer}
    \section{Waves and Sound}
    $\text{period}=\frac{\text{total time}}{\text{number of cycles}}$

    $\text{frequency}=\frac{\text{number of cycles}}{\text{total time}}$

    $v=\frac{d}{t}$

    $v=\lam\ti\frac{1}{T}$

    $v=\lam\ti f$

    $v=331.4+0.606T$

    $\mu=\frac{m}{L}$

    $v=\sqrt{\frac{F_T}{\mu}}$

    $M=\frac{\text{Speed of object}}{\text{Speed of sound in air}}=\frac{v_o}{v_s}$

    $I=\frac{P}{A}$

    $\frac{I_1}{I_2}=\frac{r_2^2}{r_1^2}$

    $\beta=10\log(\frac{I_2}{I_1})$

    $I_1=10^{-12}$

    $\frac{I_2}{I_1}=(10)^{\frac{\beta_2-\beta_1}{10}}$

    $\lam_2=\lam_1-v_oT$

    $f_2=f_1(\frac{v_s}{v_s-v_o})$

    $f_3=f_1(\frac{v_s}{v_s+v_o})$

    \section{Wave Interactions}

    $\frac{v_1}{v_2}=\frac{\lam_1}{\lam_2}$

    $L=\frac{n\lam}{2}$

    $f_n=nf_1$

    $L=\frac{(2n-1)\lam}{4}$

    $f_n=(2n-1)f_1$

    $L_2-L_1=\frac{\lam}{2}$

    $\frac{f_1}{f_2}=\frac{L_2}{L_1}=\frac{\sqrt{T_1}}{\sqrt{T_2}}=\frac{d_2}{d_1}=\frac{\sqrt{P_2}}{\sqrt{P_1}}$

    $f_B=\abs{f_2-f_1}$

    \section{Electricity}
    $e^-=-1.6\ti10^{-19}C$

    $Q=N\ti e$
    
    $I=\frac{Q}{t}$

    $V=\frac{E}{Q}$

    $E=VIt$

    $R=\frac{V}{I}$

    $R_T=R_1+R_2+R_3+\ldots$

    $V_T=V_1+V-2+V_3+\ldots$

    $I_T=I_1=I_2=I_3=\ldots$

    $\frac{1}{R_T}=\frac{1}{R_1}+\frac{1}{R_2}+\frac{1}{R_3}+\ldots$

    $I_T=I_1+I_2+I_3+\ldots$

    $V_T=V_1=V_2=V_3=\ldots$

    $P=I^2R$

    $P=\frac{V^2}{R}$

    $P=I\ti V$

    \section{Magnetism}

    $P_p=P_s$

    $V_pI_p=V_sI_s$

    $\frac{V_p}{V_s}=\frac{I_s}{I_p}$

    $V\propto N$

    $\frac{V_p}{V_s}=\frac{I_s}{I_p}=\frac{N_p}{N_s}$

    $P_{lost}=I^2R$





    % \blindtext[10]
\end{multicols}
\end{document}