\documentclass[12pt]{article}
\usepackage[utf8]{inputenc}
\usepackage{tikz}
\usepackage{graphicx}
\usepackage{siunitx} 
\usepackage{subcaption}
\usepackage{float}
\usepackage{amsmath}
\usepackage{multicol}
\usepackage{chemfig}
\usepackage{hyperref}
\usepackage{makecell}
\usepackage{isotope}
\usepackage{bohr}
\usepackage{chemfig}
\usepackage{mhchem}
% Command List
\newcommand{\sn}[2]{#1\times10^{#2}}
\newcommand{\lwd}[7]{#1[ \ce{#2} ]$^{#3}$#4[ \lewis{#5,#6} ]$^{#7}$}
\title{Chemistry Notes}
\author{Joshua Liu}
\date{\today}

\begin{document}
\maketitle
\tableofcontents'
\newpage
\section{Introduction}
Google classroom code: 5mxwn3z

\section{Scientific Notation}
Number written as the production of two numbers:
A coefficient and some power of 10
e.g $2.0\times10^{-10}$

General form of Scientific notation: $M\times10^x$ where

$M \ge 1$ but $<10$

$x$ is an integer and exponent

Determine $M$ by moving the decimal point in the original number to the left or right so that \emph{only one non-zero-digit} remains to the left of the decimal

Exercise: $1400000=\sn{1.4}{6}$

$0.000 000 000 066 7=6.67\times10^{-11}$

$5.78\times10^8=578000000$

$85000000=8.5\times10^{7}$

$0.0009=\sn{9}{-4}$

$74000=\sn{7.4}{4}$

$0.0000005=\sn{5}{-6}$

$30000000=\sn{3}{7}$

$864000=\sn{8.64}{5}$
\subsection{Adding/Subtracting}
In order to add or subtract numbers in scientific notation \textbf{your exponents must be the same}

Example: 
\begin{align*}
    &\sn{6.3}{4}+\sn{2.1}{5}\\
    &=\sn{0.63}{5}+\sn{2.1}{5}\\
    &=\sn{2.73}{5}\\
\end{align*}
\subsection{Multiplying}
When multiplying scientific notation you multiple the coefficients and \textbf{add the exponents together}

Example:
\begin{align*}
    &\sn{4}{5}\times\sn{2}{2}\\
    &=\sn{4\times2}{5+2}\\
    &=\sn{8}{7}
\end{align*}
\subsection{Division}
When dividing, divide the coefficients and \textbf{subtract the exponents}

Example:
\begin{align*}
    &(\sn{4}{5})/(\sn{2}{7})\\
    &=\sn{4/2}{5-7}\\
    &=\sn{2}{-2}\\
\end{align*}

Exercise:

$\sn{2.4}{-3}-\sn{1.2}{-2}=\sn{2.4}{-3}-\sn{12}{-3}=\sn{-9.6}{-3}$

$(\sn{7.4}{-8})/(\sn{1.2}{-2})=\sn{7.4/1.2}{-8+2}=\sn{6.2}{-6}$

$\sn{3.45}{4}\times\sn{2.3}{3}=\sn{3.45*2.3}{4+3}=\sn{7.94}{7}$

$\sn{3.6}{5}+\sn{7.82}{4}=\sn{3.6}{5}+\sn{0.782}{5}=\sn{4.4}{5}$

\section{Measurements and Significant digits}
\subsection{Accurate and Precise}
Precision: the closeness of a set of measurement of the same quantities made in the same way (how well repeated measurements of a value agrees with one another)

Accuracy: is determined by the agreement between the measured quantity and the correct value.

\href{https://www.dnasoftware.com/wp-content/uploads/2015/07/targets.png}{A good example between the difference of accuracy and precision}

\begin{tabular}{c|c}
    Precision&Accuracy\\
    \hline
    Reproducibility&Correctness\\
    Check by repeating measurements&Check by using a different method\\
    \makecell[l]{Poor precision errors from poor\\techniques}&\makecell[l]{Poor accuracy results from procedural\\or equipment flaws}
\end{tabular}

\subsection{Percentage Error}

$\text{Percentage Error}=\frac{\text{Accepted Value}-\text{Experimental Value}}{\text{Accepted Value}}\times100\%$

\subsection{Measurement}

The number of Significant digits in a value includes all digits that are certain and one that is uncertain

\subsection{Reporting Measurements}

\begin{itemize}
    \item Using Significant figures
    \item Report what is known with certainty
    \item Add one digit of uncertainty (Estimation)
\end{itemize}
\subsection{Counting Significant Figures}
When you report a measured value it is assumed that all the numbers are certain except for the last one, where there is an uncertainty of $\pm1$.

Example: the nail is 6.3 6cm long. The 6.3 are
certain values and the final 6 is uncertain! There are
3 significant figures in the value 6.36cm (2 certain
and 1 uncertain). All measured values will have one
(and one only) uncertain number (the last one) and
all others will be certain. The reader can see that the
6.3 are certain values because they appear on the
ruler, but the reader has to estimate the final 6.

\subsection{Rounding}
In all cases, round like normal except for when the number ends with only 5.

If digit before 5 is Even: you round down.

If digit before 5 is Odd: you round up.

It is a good practice to write your numbers in scientific notation to better show your significant digits

\subsection{Adding/Subtracting}
When adding and Subtracting the answer should have the \textbf{same number of decimal places} as the one with the \textbf{least number of decimal places}.

e.g $12.734-3.0=9.734->9.7$

$13.64 + 0.075 + 67=80.715->81$

$267.8 - 9.36=258.44->258.4$
\subsection{Multiplying/Dividing}
When multiplying the answer should have the same number of significant digits as the significant figure with the \textbf{least} significant digits.

e.g $61\times0.00745=0.45445=\sn{4.5}{-1}$

$608.3\times3.45=2098.635=\sn{2.10}{3}$

$4.8\div392=0.012245=\sn{1.2}{-2}$

\section{Lab Safety}
\begin{enumerate}
    \item Read and follow all directions exactly as they are written. If in doubt, ask your teacher.
    \item Never mix chemicals (or perform tests) without your teacher's permission.
    \item Keep your work area clean and keep all materials (clothing, hair, papers, ect.) away from a flame or heat source.
    \item Never run or push someone else in the lab. This rule applies at all times.
    \item Always wear safety goggles whenever you are working with chemicals or other substances that might get into your eyes.
    \item Immediately notify your teacher if any chemicals gets on your skin or clothing to find out what to do to clean it off.
    \item Immediately notify your teacher if you get cut or have another injury when performing an experiment.
    \item Never reach across a flame.
    \item Never look directly into a test tube when mixing or heating chemicals.
    \item Always point a test tube away from you and others when heating over a flame or other heat source.
    \item Never smell a chemical directly from the container. Wave your hand over the opening of the container and "waft" the fumes toward your nose.
    \item Never taste a chemical unless you are instructed by your teacher to do so.
    \item Never use broken or chipped glassware.
    \item Keep lids on bottles and containers when not in use.
    \item Wash your hands before and after each experiment.
    \item Always clean up your work area and equipment after an experiment us completed. Equipment must be returned to its proper place.
\end{enumerate}
\section{Atom and Periodic Table}

\begin{tabular}{c|c|c|c}
    Name&Relative Mass&Electric Charge&Location\\
    \hline
    Proton&1&+&Nucleus\\
    Neutron&1&0&Nucleus\\
    Electron&1/10000&-&Outside the Nucleus\\
\end{tabular}
\subsection{Atomic Math}

\textbf{Atomic Number}: Number of protons/electrons

\textbf{Atomic Mass}: \# of protons + \# of neutrons
\subsection{Representing Elements}

Standard Notation: 
\isotope[22.990][11]{Na}

\subsection{Isotopes}

\textbf{Isotopes}: atoms of the same element with different numbers of neutrons

\subsection{Bhor-Rutherford Diagram}

Put number of protons and neutrons in the middle

Draw every electrons

e.g \href{https://d2jmvrsizmvf4x.cloudfront.net/BOV8bU6eSfuqDVGVKVbg_atomic_structure_of_sodium.jpg}{Sodium}

\subsection{Lewis Diagram}

Lewis is simpler than Bhor-Rutherford

Put symbol of the element in the centre

Add valence shell around the symbol

e.g Sodium

\lewis{0:2:4.6.,Na}

\section{Isotopes}

Isotopes-Atoms that have the \emph{same} number of protons but \emph{different} number of neutrons are called isotopes.

Isotopes have different mass number.

\subsection{Average Atomic Mass}

Most elements occur naturally as mixtures of isotopes.

The mass number on the periodic table are the weighted average of the most abundant isotopes mass numbers

The atomic mass of an element is a weighted average mass of the atoms in a naturally occurring sample of the element

%TODO: Copied directly from slides Trim down to be concise

\subsection{Radioisotopes}
Some isotopes are stable, others break apart easily.

Difference in stability is due to the composition of the nuclei

\textbf{Unstable} isotopes emit nuclear radiation are known radioisotopes.

When a radioactive isotope breaks apart we get \textbf{Radioactive Decay}

Three types

\begin{itemize}
    \item Alpha particles
    \item Beta particles
    \item Gamma rays
\end{itemize}


\textbf{Alpha Decay}: loses two protons and two neutrons this particle is called Alpha particle. $\ce{^4_2He}$

$\ce{^{226}_{88}Ra -> ^{222}_{86}Rn + ^4_2He}$

\textbf{Beta Decay}: loses one electron this particle is called Beta particles. $\ce{^0_{-1}\beta}$

$\ce{^{14}_6C -> ^{14}_7N + ^0_{-1}\beta}$

\textbf{Gamma Decay}: loses no mass or charge but loses energy this is called Gamma particles. $\ce{^0_0\gamma}$

$\ce{^{60}_{27}Co -> ^{60}_{27}Co + ^0_0\gamma}$

In terms of size, Alpha particles are the largest, Beta particles are the second, and Gamma particles are the smallest.

Alpha particles can't pass through paper

Beta particles can't pass through aluminum

Gamma particles can partially pass through lead

\subsection{Calculating Average Atomic Mass}

\textit{Basic math, I don't see a reason to write about this}

\section{The Modern Periodic Table}

\textbf{Groups}: vertical columns (1-18)

\noindent\textbf{Periodic}: horizontal rows (1-7)

\noindent\textbf{Periodicity}: The similarities of the elements in the same group is explained by the arrangement of valence electrons

\noindent\textbf{Atomic Radius}: is the distance from the centre of the nucleus of an atom to the outermost electron.

The greater the number of energy levels the greater is the distance of the outermost electron to the centre of it atom's nucleus.

\medskip

The size of the atomic radius cannot be measured exactly because it does not have a sharply defined boundary. The atomic radius is measured by the distance of the centre of two nuclei of two atoms beside each other divided by 2.

\noindent\textbf{Trends in atomic radii}: atomic radii decreases as you move up in a group, and decreases as you move across a period.

\medskip

\noindent\textbf{Force of attraction}: The attraction of the electrons to the nucleus is what keeps the electrons with the nucleus.

\noindent There are two factors that affect the force of attraction

\medskip
The size of the positive charge determined by the number of protons.

The distance between the outermost electron and the nucleus

\noindent \emph{The balance exists between the attraction of the electron to the nucleus and the repulsion between the electrons themselves.}

\medskip
\noindent The valence electrons receive a positive charge from the nucleus as the inner electrons weaken the attraction for the valence electrons this is called \textbf{Shielding Effect}

\subsection{Effective Nuclear Charge}

A number assigned to elements to describe the amount of shielding to valence electrons

$$\text{ENC}=\text{\# of Protons}-\text{\# of Inner Electrons}$$

The greater the ENC the stronger the attraction to valence electrons

The greater the ENC the smaller atomic radius

\subsection{Atomic Radius of Ions}

Sodium: $\ce{Na}$

$$ENC=11-10=1$$

\noindent Sodium ion: $\ce{Na^+}$

$$ENC = 11-2=9$$

\noindent Cations will have smaller ionic radius that neutral atom.

\noindent Anions will have a larger ionic radius than the neutral atom.

\subsection{Ionization Energy}

The amount of energy to remove the valence electron to form an ion. 

\href{http://www.vias.org/genchem/atomstruct_12433_05.html}{Refer to first graph}

\medskip
\noindent Moving up the group, it takes more energy.

\noindent Moving up the period, it takes more energy.

\subsection{Successive Ionization Energy}

After the first Ionization Energy (IE) is removed, the successive IE increases as it becomes more difficult to move the next electrons since the pull from the nucleus becomes stronger.

\medskip
\noindent There's a huge jump in IE between removing the last valence electron and removing the first electron in the new ring.

\subsection{Electron Affinity}

Electron Affinity is the amount of energy released when an electron is captured by an atom to form a negative ion (anion).

$$\ce{Cl +\text{Electron} -> Cl^- + \text{Energy}}$$

\medskip

\noindent \textbf{Periodic Trend}: Electron Affinity increases as you move up the periods.

\noindent\textbf{Group Trend}: Electron Affinity increases as you move to the right of the group.

\medskip
\noindent Electron Affinity and Electronegativity follow the same general Trend.

\emph{Halogens are the peaks while noble gases are the troughs.}

\subsection{Electronegativity}

Electronegativity is the tendency an electron would be attracted to a atom when combining with another element.

\emph{The scale is does not have a value like km/h but is an arbitrary value between 0 and 4.}

\noindent In general, \textbf{metals} have a \textbf{low} EN while \textbf{nonmetals} have a \textbf{high} EN.

\medskip
\textit{Fun fact: Fluorine has the highest EN at 4.}

\medskip

\noindent Refer to the trends of Electron Affinity trend for trend of Electronegativity.

\subsection{Reactivity}

When looking at \textbf{nonmetals}, they lose electrons and the reactivity trend follows the trend of \emph{Electron Affinity}

\medskip
\noindent When looking at \text{metals}, they gain electrons and the reactivity trend follows the trend of \emph{Ionization Energy}

\subsection{Summary}

\renewcommand{\arraystretch}{1.25}
\makebox[1 \textwidth][c]{
    \begin{scriptsize}
        \begin{tabular}{c|c|c|c}
            Name&Definition&Group Trend going down&Periodic Trend going left to right\\
            \hline
            Atomic Radius&\makecell[c]{Distance from centre of nucleus\\to outermost electron (pm)}&Increases&Decreases\\
            First IE&Energy to remove outmost electron&Decreases&Increases\\
            Electron Affinity&Energy released when gaining electron&Decreases&Increases excluding group 18\\
            Electronegativity&Tendency to gain electrons&Decreases&Increases excluding group 18\\
            Reactivity Metal&Degree to which metal react&Increases&Decreases\\
            Reactivity Nonmetals&Degree to which nonmetal react&Decreases&Increases excluding group 18\\
        \end{tabular}
    \end{scriptsize}
}
\renewcommand{\arraystretch}{1}

\section{Chemical Bonding}

All atoms are  trying to achieve a \textbf{stable octet}

\medskip
\noindent The proton in one nucleus are attracted to the electron of another nucleus based of their electronegativity

\noindent Ionic bonding: Forms ionic compounds through atoms giving and taking electrons

\noindent Covalent bonding: Forms molecules through atoms by sharing electrons

\noindent Metallic bonding: Creates a an excess of electrons as both are positive ions (Mobile Electrons). Allows it to conduct electricity.

\medskip
\noindent Metallic Characteristics:

\begin{itemize}
    \item High melting point, ductile, malleable, shiny
    \item Hard substance
    \item Good conductor as solid and liquid
\end{itemize}

\subsection{Ionic Bonding}

Electrons are \textbf{transferred} between valence shells of atoms.

\noindent Ionic compounds are made of ions

They are called salts or crystals 

\medskip
\noindent \textbf{Always} forms between metals and non-metals

Ionic compounds formed a difference in electronegativity of 1.7 or greater

\medskip
\noindent Polyatomic ions always form ionic compounds

Properties

\begin{itemize}
    \item Hard solids are 22$^{\circ}$C.
    \item High melting point
    \item Non-conductive as solids
    \item Good conductors as liquids or when dissolved in water (aq)
\end{itemize}

\subsection{Covalent Bonding}
Electrons are \textbf{shared} between non-metal atoms.

Electronegativity difference less than 1.7

\medskip
\noindent Covalent bonding forms (two or more) polyatomic ions

Properties
\begin{itemize}
    \item Low melting point, low boiling point
    \item softer solid compared to ionic compounds
    \item Non conductors, period.
\end{itemize}

\subsection{Drawing ionic compounds}

\begin{large}
    Lewis Dot Diagram
\end{large}

\medskip
\noindent Represents the valence electrons as dots

e.g: NaCl

\begin{center}
    [Na]$^+$[ \lewis{0:2:4:6:,Cl} ]$^-$
\end{center}

e.g: AlS

\begin{center}
    2[Al]$^{3+}$ 3[ \lewis{0:2:4:6:,S} ]$^{2-}$
\end{center}

Practice
\medskip

\renewcommand{\arraystretch}{1.25}
\begin{center}
\begin{tabular}{c|c}
    \ce{LiF}&\lwd{}{Li}{+}{}{0:2:4:6:}{F}{-}\\
    \ce{MgO}&\lwd{}{Mg}{2+}{}{0:2:4:6:}{O}{2-}\\
    \ce{CaCl2}&\lwd{}{Cl}{2+}{2}{0:2:4:6:}{Cl}{-}\\
    \ce{K2S}&\lwd{2}{K}{+}{}{0:2:4:6:}{S}{2-}\\
\end{tabular}
\end{center}
\renewcommand{\arraystretch}{1}
\subsection{Drawing Molecules}

\textit{I don't do these but they might help}

Steps
\begin{enumerate}
    \item Count total valence electrons involved
    \item Connect the central atom (usually the first in the formula) to the other with single bonds
    \item Complete valence shells of outer atoms
    \item Add any extra electrons to central atom
\end{enumerate}



\end{document}