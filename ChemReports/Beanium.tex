\documentclass[10pt]{extarticle}
\usepackage[utf8]{inputenc}
\usepackage{cite}
\usepackage{array}
\usepackage{amsmath}
\title{Average Atomic Mass Lab "BEANIUM"}
\author{Joshua Liu}
\date{\today}

\renewcommand{\arraystretch}{1.5}


\begin{document}
    \maketitle
    After following the procedures given, we created an observation table
    
    \begin{center}
        \begin{tabular}{c|c|c|c}
            Isotope Name&\# of Atoms&Total Mass (g)&Qualitative Observations\\
            \hline
            Rednium&52&33.99&Length of fingernail, Rounded, Smooth, Red\\
            Brownium&106&36.64&Spotted, Round, Oblong, Brown, Smooth\\
            Whitenium&220&43.33&Length of pinky, Rippled surface, White\\
        \end{tabular}
    \end{center}

    Total Atoms: 378
    \section*{Analysis}

    1. How many isotopes does Beanium have?

    Beanium has 3 different isotopes, Rednium, Brownium, and Whitenium.

    \medskip
    \noindent 2. Using the total mass of each isotope and the number of atoms for each isotope, determine the average mass for each individual isotope.

    \medskip
    \begin{tabular}{c|c|c|c}
        &Rednuim&Brownium&Whitenium\\
        \hline
        \newline
        \vspace*{2 cm}
        \newline
        Average Mass (g)&$\frac{33.99}{52}=0.654$&$\frac{36.64}{106}=0.346$&$\frac{43.33}{220}=0.197$\\
    \end{tabular}
    
    \noindent 3. Calculate the relative abundance of each isotope

    \noindent Rednium:
    \begin{align*}
        \text{Relative Abundance}&=\frac{\text{\# of Beans}}{\text{Total \# of Beans}}\times100\%\\    
        &=\frac{52}{378}\times100\%\\
        &=13.76\%\\
    \end{align*}
    \newpage
    \noindent Brownium:
    \begin{align*}
        \text{Relative Abundance}&=\frac{106}{378}\times100\%\\
        &=28.04\%\\
    \end{align*}
    \noindent Whitenium
    \begin{align*}
        \text{Relative Abundance}&=\frac{220}{378}\times100\%\\
        &=58.20\%\\
    \end{align*}
    \noindent 4. Determine the average atomic mass of Beanium
    \medskip
    $$\text{Average Atomic Mass}=0.654*13.76\%+0.346*28.04\%+0.197*58.20\%$$
    $$=0.302\text{ g}$$

    \noindent The average atomic mass of Beanium is 0.302 grams.
\end{document}