\documentclass[10pt]{article}
\usepackage{tabularx}
\usepackage{chemfig}
\usepackage{mhchem}
\usepackage{ifthen}
\usepackage{amsmath}

\usepackage{ulem}
\usepackage{amssymb}
\usepackage[margin=1in]{geometry}


\title{Quantities Package}
\author{Joshua Liu}
\date{\today}

\begin{document}
\maketitle

\begin{enumerate}
    \item Calculate the following
    \begin{enumerate}
        \item Calculate the mass of \ce{Cl2} that combines with 2.36 g of \sout{\ce{Cl2}} \textit{(I'm assuming this was an error and is \ce{H2})} to form \ce{HCl} in the following reaction
        
        \ce{H2_{(g)} + Cl2_{(g)} -> 2HCl_{(g)}}

        \bigskip
        $$2.36\text{ g }\ce{H2}*\frac{1\text{ mol }\ce{H2}}{2*1.01\text{ g }\ce{H2}}*\frac{1\text{ mol }\ce{Cl2}}{1\text{ mol }\ce{H2}}*\frac{2*35.45\text{ g }\ce{Cl2}}{1\text{ mol }\ce{Cl2}}$$

        $$=82.83\text{ g }\ce{Cl2}$$

        $\therefore$ The mass required of \ce{Cl2} to combine with 2.36 g of \ce{H2} is 82.83 g.
        \bigskip
        \item What mass of \ce{Al(OH)3} is formed when 7.5 g of \ce{H2O} reacts with excess of \ce{AlP}? The balanced chemical equation is
        
        \ce{AlP + 3H2O -> PH3 + Al(OH)3}
        \bigskip
        $$7.5\text{ g }\ce{H2O}*\frac{1\text{ mol }\ce{H2O}}{2*1.01+16\text{ g }\ce{H2O}}*\frac{1\text{ mol }\ce{Al(OH)3}}{3\text{ mol }\ce{H2O}}*\frac{26.98+(16+1.01)*3\text{ g }\ce{Al(OH)3}}{1\text{ mol }\ce{Al(OH)3}}$$

        $$=10.82\text{ g }\ce{Al(OH)3}$$

        $\therefore$ 10.82 g of \ce{Al(OH)3} is formed with 2.36 g of \ce{H2} and excess of \ce{AlP}.
    \end{enumerate}
    \bigskip
    \item Given the following equation
    
    \ce{Al2(SO3)3 + 6 NaOH -> 3Na2SO3 + 2 Al(OH)3}
    \begin{enumerate}
        \item If you start with 16 g of \ce{Al2(SO3)3} and 12.5 g of \ce{NaOH}, determine the limiting reagent.
        \medskip

        What we will do is choose an arbitrary compound say, \ce{Al2(SO3)3} and calculate the mass needed to completely react with \ce{Al2(SO3)3} if it's less than the given mass of \ce{NaOH}, then \ce{Al2(SO3)3} is the limiting reagent, otherwise \ce{NaOH} is the limiting reagent.
        \medskip

        Molar Mass of \ce{Al2(SO3)3} = $26.09*2+(32.07+16*3)*3=292.39$ g/mol

        Molar Mass of \ce{NaOH} = $22.99+16.00+1.01=40.00$ g/mol
        $$16\text{ g }\ce{Al2(SO3)3}*\frac{1\text{ mol }\ce{Al2(SO3)3}}{292.39\text{ g }\ce{Al2(SO3)3}}*\frac{6\text{ mol }\ce{NaOH}}{1\text{ mol }\ce{Al2(SO3)3}}*\frac{40.00\text{ g }\ce{NaOH}}{1\text{ mol }\ce{NaOH}}$$

        $$=13.13\text{ g }\ce{NaOH}$$

        $\therefore$ \ce{NaOH} is the limiting factor as the required mass for a full reaction with \ce{Al2(SO3)3} is greater than the given mass.
        \bigskip
        \item Determine the mass of \ce{Na2SO3} produced.
        \medskip

        As \ce{NaOH} is the limiting reagent, we must calculate the mass of \ce{Na2SO3} with \ce{NaOH}
        \medskip

        Molar Mass of \ce{Na2SO3} = $2*22.99+32.07+3*16.00=126.05$ g/mol

        $$12.5\text{ g }\ce{NaOH}*\frac{1\text{ mol }\ce{NaOH}}{40.00\text{ g }\ce{NaOH}}*\frac{3\text{ mol }\ce{Na2SO3}}{6\text{ mol }\ce{NaOH}}*\frac{126.05\text{ g }\ce{Na2SO3}}{1\text{ mol }\ce{Na2SO3}}$$

        $$=19.70\text{ g }\ce{Na2SO3}$$

        $\therefore$ The mass of \ce{Na2SO3} produced is 19.70 grams.
        \bigskip
        \item Determine the mass of the excess reagent left over.
        \medskip

        To calculate the mass of the excess reagent, we can calculate the mass used in the reaction with the limiting reagent.

        $$12.5\text{ g }\ce{NaOH}*\frac{1\text{ mol }\ce{NaOH}}{40.00\text{ g }\ce{NaOH}}*\frac{1\text{ mol }\ce{Al2(SO3)3}}{6\text{ mol }\ce{NaOH}}*\frac{292.39\text{ g }\ce{Al2(SO3)3}}{1\text{ mol }\ce{NaOH}}$$

        $$=15.23\text{ g }\ce{Al2(SO3)3}$$

        Mass of excess reagent left over = $16-15.23=0.77\text{ g }\ce{Al2(SO3)3}$

        $\therefore$ The mass of excess reagent left over is 0.77 g of \ce{Al2(SO3)3}
        \bigskip
        \item What is the percentage yield if 4.45 g of \ce{Na2SO3} was actually produced? 
        
        $$\text{Percentage Yield}=\frac{\text{Actual Yield}}{\text{Theoretical Yield}}*100\%$$

        $$=\frac{4.45\text{ g }\ce{Na2SO3}}{19.70\text{ g }\ce{Na2SO3}}*100\%$$

        $$=23\%$$

        $\therefore$ The percentage yield is 23\% if 4.45 g of \ce{Na2SO3} was actually produced.
    \end{enumerate}
    \item A student conducts and experiment involving the following balanced chemical equation
    
    \ce{H3PO4 + 3KOH -> K3PO4 + 3H2O}

    What is the percentage yield if the student obtains 49.0 g of \ce{K3PO4} when she combines 67.0 g of \ce{H3PO4} with 62.0 g of \ce{KOH}?

    \bigskip

    Molar Mass of \ce{H3PO4} = $3*1.01+30.97+4*16.00=98.00$ g/mol

    Molar Mass of \ce{KOH} = $39.10+16.00+1.01=56.11$ g/mol

    Molar Mass of \ce{K3PO4} = $3*39.10+30.97+4*16.00=212.27$ g/mol
    \medskip

    We must first find the limiting factor, then we can use the the limiting reagent to calculate the theoretical yield, which will be used to calculate the percentage yield.

    Using a different method than 2a), we can calculate the limiting factor by calculating the mole of each compound, and whichever is less is the limiting factor.
    
    $$67.0\text{ g }\ce{H3PO4}*\frac{1\text{ mol }\ce{H3PO4}}{98.00\text{ g }\ce{H3PO4}}*\frac{1\text{ mol }\ce{K3PO4}}{1\text{ mol }\ce{H3PO4}}=0.68\text{ mol }\ce{K3PO4}$$

    $$62.0\text{ g }\ce{KOH}*\frac{1\text{ mol }\ce{KOH}}{56.11\text{ g }\ce{KOH}}*\frac{1\text{ mol }\ce{K3PO4}}{3\text{ mol }\ce{KOH}}=0.37\text{ mol }\ce{K3PO4}$$

    $\therefore$ The limiting reagent is KOH and we will calculate the theoretical yield with KOH

    As we already know 62 g of \ce{KOH} is 0.37 mol of \ce{K3PO4} we will continue from there.

    $$0.37\text{ mol }\ce{K3PO4}*\frac{212.28\text{ g }\ce{K3PO4}}{1\text{ mol }\ce{K3PO4}}=78.54\text{ g }\ce{K3PO4}$$

    $\therefore$ The theoretical yield is 78.54 g of \ce{K3PO4}

    $$\text{Percentage Yield}=\frac{\text{Actual Yield}}{\text{Theoretical Yield}}*100\%$$

    $$=\frac{49.0\text{ g }\ce{K3PO4}}{78.54\text{ g }\ce{K3PO4}}*100\%$$

    $$=62\%$$

    $\therefore$ The student's percentage yield is 62.
\end{enumerate}

\end{document}
