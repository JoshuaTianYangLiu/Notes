\documentclass[../main.tex]{subfiles}


\begin{document}
\section{Chapter 1}
\subsection{Domain and range}
$$y=3x-5$$
Domain=$\{x\in\!R\}$
Range=$\{y\in\!R\}$

$y=x^2$ is a function because it passes the vertical line text

\noindent$x=y^2$ is not a function because it does not pass vertical line test.

\subsection{Absolute Value}

$$|-22|=22$$

$$|10-8|=|-8|=|8|$$

Suppose $|x|<8$ is true which means $-8<x<8$

\subsection{Transformations}

$$y=\frac{1}{2}f(\frac{1}{4}(x-5))+6$$

Vertical compression by a factor of $\frac{1}{2}$

Horizontal stretch by a factor of $4$

Horizontal translation $5$ units to the right

Vertical translation $6$ units up

$y=\sqrt{x}$ to $y=-3\sqrt{x-5}$ makes a mapping of $(x,y)\rightarrow(x+5,-3y)$

Domain=$\{x\in\!R|x\ge5\}$
Range=$\{y\in\!R|x\le0\}$

\begin{tabular}{c|c}
    $x$&$y$\\
    \hline
    $0$&$0$\\
    $1$&$1$\\
    $4$&$2$\\
    $9$&$3$\\
\end{tabular}
\quad
\begin{tabular}{c|c}
    $x$&$y$\\
    \hline
    $5$&$0$\\
    $6$&$-3$\\
    $9$&$-6$\\
    $14$&$-9$\\
\end{tabular}
% put on same line

Transformations follow the format $y=a\times f(b(x-c))+d$ Noticed the $-c$ and the $+d$, they are set that way so when you list the transformations they follow a patter

Vertical stretch by a factor of a

Horizontal stretch by a factor of $\frac{1}{b}$

Horizontal translation $c$ units to the right

Vertical translation $d$ units up

Make sure the translations are listed as if you're reading the function from the left to right, it is the correct way.

Transformations are the way you describe the transformation say $y=x-5$ you say \textit{Horizontal translation 5 units right} the mapping should follow the statement so $(x,y)\rightarrow(x+5,y)$
\subsection{Inverse Function}
\begin{center}
    $$(2,5)\rightarrow(5,2)$$
    $$(4,-8)\rightarrow(-8,4)$$
    $$f(1)=2$$
    $$f(2)^{-1}=1$$
\end{center}
\quad
\begin{center}
    $$f(x)=3x+4$$
    $$y=3x+4$$
    $$x=3y+4$$
    $$x-4=3y$$
    $$\frac{x-4}{3}=y$$
    $$f(x)^{-1}=\frac{x-4}{3}$$
\end{center}
\quad
\begin{center}
    $$g(x)=4(x-3)^2+1$$
    $$x=4(y-3)^2+1$$
    $$\frac{x-1}{4}$$
    $$\pm\sqrt{\frac{x-1}{4}}+3=y$$
    $$f(x)^{-1}=3\pm\sqrt{\frac{x-1}{4}}$$
\end{center}

The idea of inverse functions is to think of the function as $y=mx+b$ or $y=ax^2+bx+c$, where the $y$ is on the left and the rest is on the right, calculating the inverse is just isolating for $x$ then replacing $y$ with $f(x)^{-1}$ and x with $x$, the process shown above is a more formal way to show it.
\subsection{Piecewise Functions}

I do not want to draw graphs...

This chapter is about putting together functions, knowing how to graph them, how to get the function from the graph, and how to adjust so that functions are continuous

\subsection{Operations with Functions}

When it gives you a two sets and tell you to add, subtract, multiply, or whatever you take the y values and you perform the operation on those values. Make sure your x values are the same

$f=\{(0,1)\}$ $g=\{(3,2)\}$ $f+g$ This does not work as the x values are different.

$f=\{(0,1)\}$ $g=\{(0,2)\}$ $f-g$ This would equal to $\{(0,-1)\}$

The rest is just adding functions

\section{Chapter 3}
\subsection{Polynomials}

$f(x)=a_nx^n+a_{n-1}x^{n-1}+\ldots+a_0$

\subsection{Characteristics of Polynomials}

Polynomials of an even degree have both end points going in the same direction
Polynomials of an odd degree have both end points going in different directions

 

\end{document}