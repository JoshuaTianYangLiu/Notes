\documentclass[../main.tex]{subfiles}


\begin{document}
\subsection{Polynomials}

Say we have a factored polynomial which follows $y=a(x-x_1)(x-x_2)(x-x_3)^3$ it could also be $y=a(x-x_1)(x-x_2)^2$ We can use certain characteristics of the factored form to draw a graph, looking at each factor ($(x-x_n)$) if it has a power (order) of 1 it passes through that x axis

\noindent If it is order 2 it touched the point at the x axis and "bounces off of it"

\noindent Order 3 it changes "concavity" (You will learn this later)

say $y=2(x+1)^2(x-3)$, at $(0,-1)$ it will bounce off the axis, and at $(x-3)$ it will go through that point

View the graph on \hyperlink{https://www.desmos.com/calculator/zn0lesjogb}{Desmos}.

You can also take this idea to find the rough functions of graphs.

Turning points are points in the polynomial which "turn" the way it's facing, better known as maximum and minimum, think of it as the vertices of the polynomial

\subsection{Transformations of cubic and quartic functions}

TLDR: Just transformations of function cubic and quartic functions are involved, (Really just transformation of functions)

\subsection{Dividing Polynomials}

Remember long division is middle school? \longdiv{3}{253} dividing polynomials has the same idea,

$(x^3-2x+2)/(x-4)$

\polylongdiv{x^3+0x^2-2x+2}{x-4}

Dividing by this we get $x^2+4x+14$ remainder $58$, written out is $x^2+4x+14+\frac{58}{x-4}$

There is also synthetic division it's takes less space, and only works in cases where you have $x-x_1$. I suggest long division, it's like a tool which does it all while synthetic only has one use case, but a common one.


\noindent Remainder theorem

say you have a polynomial divided by a linear, to find the remainder you solve for x in the linear function and plug that into the polynomial

$$(x^3-2x+2)/(x-4)$$

$$x-4=0 \rightarrow x=4$$

$$4^3-2*4+2=58$$

\textbf{This only works when dividing by linear function}

\noindent You can use this theorem to solve some problems 

$8x^3+10x^2-px-5$ is divisible by $2x+1$

You can use the remainder theorem 

$$2x+1=0 \rightarrow x=-\frac{1}{2}$$

$$P(x)=8x^3+10x^2-px-5$$

$$P(-\frac{1}{2})=\frac{1}{2}p-3=0$$

$$p=6$$

You can also use long division to find it but it will take long (I'd still do it tho :>)

\subsection{Factoring Polynomials}

So you got a formula for factoring quadratic polynomials, what about cubic? quartic? there is a formula for cubic but it's scary.

So what we can do is try to find a factor in a polynomial by brute forcing for a factor.

Say factor $x^3+2x^2-5x-6$, using the factor theorem before try plugging in values and see if you can find one which equals $0$. Through brute force we find $P(-1)=0$ meaning $x+1$ is a factor of the polynomial

\polylongdiv{x^3+2x^2-5x-6}{x+1}

Knowing the factor and the quadratic part we can easily factor the quadratic to get the factored polynomial $(x+1)(x+3)(x-2)$

\subsection{Difference Of Cubes}

Difference of cubes $a^3-b^3=(a-b)(a^2+ab+b^2)$

\noindent Sum of cubes $a^3+b^3=(a^2-ab+b^2)$

\subsection{Solving Polynomials}

$$y=2x(x-1)(x+2)(x-2)$$
$$x=0,1,-2,2$$

I mean, this is grade 11 polynomials.

\end{document}