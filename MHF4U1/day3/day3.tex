\documentclass[../main.tex]{subfiles}


\begin{document}
\section{Chapter 4}
\subsection{Polynomial Inequalities}

$(x+2)(x-3)(x+1)\ge0$ Graphing this function you get two intervals which satisfy this condition, $-2\le x\le -1$ or $x\ge3$


\subsection{Rate Of Change}

Rate of change $m=\frac{y_2-y_1}{x_2-x_1}$ you should already know this from middle school. Use to find slopes of functions at a and b


We can use this to estimate the tangent at any point in a function by finding the slope with two points that are very close together.

% No way chapter 5 is that short

\section{Chapter 5}
\subsection{Rational Functions}

Say we want to graph $y=\frac{1}{x-5}$ there are two "lines" we can draw that describes the function will never pass through, where they go to and from infinity at that point, we call those "asymptotes", for example, that function has two asymptotes, $x=5$ and $y=0$ 

Say we have a quadratic ration function $y=\frac{1}{x^2-x-6}$ when we graph $x^2-x-6$ we know the x intercepts are $-2$ and $3$ and we know $\frac{1}{0}$ is undefined so these two points are asymptotes $x=-2$ and $x=3$ there is also a asymptote at $y=0$. Domain = $\{x\in\!R|x\ne -2,3\}$, Range = $\{y\in\!R | y<-\frac{1}{2} \text{ or } y>0\}$

\subsection{Graph Rational Functions}

So how do we graph something like $y=\frac{ax+b}{cx+d}$

$y=\frac{x+4}{2x+5}$ We have a horizontal asymptote at $y=\frac{a}{c}$ or $y=\frac{1}{2}$ and a vertical asymptote $2x+5=0 \rightarrow x=-\frac{5}{2}$ 

These functions can also have a function which can be a "hole" say $y=\frac{x+2}{2x+4}$ if we were to calculate it's "vertical asymptote" $2x+4=0\rightarrow x=-2$ if we also calculate the numerator at this point we also get $0$ meaning we get $\frac{0}{0}$, this means it is a hole instead of a vertical asymptotes. hole at $(-2,\frac{1}{2})$ and a horizontal asymptote at $y=-\frac{1}{2}$

% I kinda skipped a bit, just some stuff about ration equations
Solve the inequality $x(x-5)^2(x+6)^2>0$ 

Zeros are $x=0,5,-6$. We calculate the values between the intervals of $-6,0,5$ and noticed $x<-6$ we get a positive number, $-6<x<0$ negative $0<x<5$ positive $5<x$ positive.

Meaning at $x<-6,0<x<5,x>5$ is rings true.

\noindent Find a rational function with a hole at $x=1$ vertical asymptotes at $x=5$ and horizontal asymptote at $y=-8$

$$\frac{-8x(x-1)}{(x-1)(x-5)}$$

$x-5$ means vertical asymptote, both $x-1$ represent the hole, and $-8x$ and $x-5$ represent vertical asymptote.

To find the end behavior (when x approaches negative and positive infinity, usually) of a rational function, you take a look at it's horizontal asymptote. 

$f(x)=\frac{6x-5}{3x+12}$

H.A. $y=2$. V.A. $x=-4$

\begin{enumerate}
    \item As $x\rightarrow -\infty$, $y\rightarrow2$ from above
    \item As $x\rightarrow +\infty$, $y\rightarrow2$ from below
    \item As $x\rightarrow-4^-$, $y\rightarrow+\infty$
    \item As $x\rightarrow-4^+$, $y\rightarrow-\infty$
\end{enumerate}

\subsection{Solving Rational Inequalities}

Simplify and factor, check the zeros of both the numerator and denominator. Check the intervals between zeros 
\end{document}